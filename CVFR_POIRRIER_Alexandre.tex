\documentclass[a4paper,icon]{twentysecondcv} % a4paper for A4

\usepackage[french]{babel}
% Linkedin
\newcommand{\givencvlinkedin}{}
\newcommand{\cvlinkedin}[1]{\renewcommand{\givencvlinkedin}{#1}}

% Skills
\renewcommand{\skills}[1]{\renewcommand{\givenskill}{#1}}
\newcommand{\skillsection}[1]{~\\[.2cm]{\noindent\large #1}}

% Enable French
\newcommand{\skillheader}{\profilesection{Skill}{5cm}}
\iflanguage{french}{\renewcommand{\skillheader}{\profilesection{Compétences}{1.9cm}}}{}
\newcommand{\aboutmeheader}{\profilesection{About me}{3.2cm}}
\iflanguage{french}{\renewcommand{\aboutmeheader}{\profilesection{\`A propos}{3.4cm}}}{}

%% Profile section
\newgeometry{left=8.5cm,top=0.1cm,right=1cm,bottom=0.2cm,nohead,nofoot}
\renewcommand{\makeprofile}
{
  % grey bar on the left side
  \begin{tikzpicture}[remember picture,overlay]
      \node [rectangle, fill=asidecolor, anchor=north, minimum width=9.90cm, minimum height=\paperheight+1cm] (box) at (-5cm,0.5cm){};
  \end{tikzpicture}

  \begin{textblock}{7.5}(0.5, 0.2)
    \begin{flushleft}
      \hspace{13pt}

      \begin{center}
        % profile picture
        \ifthenelse{\equal{\givenprofilepic}{}}{}{\begin{tikzpicture}[x=\imagescale,y=-\imagescale]
            \clip (600/2, 567/2) circle (567/2);
            \node[anchor=north west, inner sep=0pt, outer sep=0pt] at (0,0) {\includegraphics[width=\imagewidth]{\givenprofilepic}};
        \end{tikzpicture}}

        % name
        {\Huge\color{mainblue}\givencvname}
      \end{center}

      % jobtitle
      \begin{flushright}
        {\Large\color{black!80}\givencvjobtitle}
      \end{flushright}
      \renewcommand{\arraystretch}{1.6}

      % table with icons 
      \begin{tabular}{c @{\hskip 0.2cm} p{5cm}}
        % CV date
        \ifthenelse{\equal{\givencvdate}{}}{}{\textsc{\Large\icon{\Info}} & \givencvdate\\}
        % CV address
        \ifthenelse{\equal{\givencvaddress}{}}{}{\textsc{\Large\icon{\Letter}} & \givencvaddress\\}
        % CV phone
        \ifthenelse{\equal{\givennumberphone}{}}{}{\textsc{\Large\icon{\Telefon}} & \givennumberphone\\}
        % CV site
        \ifthenelse{\equal{\givencvsite}{}}{}{\textsc{\Large\icon{\Mundus}} & \href{\givencvsite}{\textcolor{cerulean}\givencvsite}\\}
        % CV mail
        \ifthenelse{\equal{\givencvmail}{}}{}{\textsc{\large\icon{@}} & \href{mailto:\givencvmail}{\givencvmail}\\}
        % CV linkedin
        \ifthenelse{\equal{\givencvlinkedin}{}}{}{\textsc{\large\icon{\faLinkedin}} & \href{\givencvlinkedin}{\textcolor{cerulean}\givencvlinkedin}}
      \end{tabular}

      % about me text
      \aboutmeheader~\\
      \givenaboutme

      % skills with scale
      \skillheader
      \givenskill
    \end{flushleft}
  \end{textblock}
  \vspace{-10pt}
}

%% Section style
\renewcommand{\section}[1]
{
  \par%
  {%
    \LARGE
    \color{headercolor}%
    \round{#1}{%
      \ifodd\value{colorCounter}%
        mainblue%
      \else%
        maingray%
      \fi%
    }%
  }
  \stepcounter{colorCounter}%
  \par\vspace{\parskip}
}

\renewcommand{\twentyitem}[4]{
  \begin{tabular}{l @{\hskip 0.2cm} p{9cm} @{\hskip 0.2cm} r}
    #1 & #2 & \footnotesize#3 \\
    & #4 & 
  \end{tabular}
  ~\\\vspace{\parsep}
}

\usetikzlibrary{math}
%----------------------------------------------------------------------------------------
%	 PERSONAL INFORMATION
%----------------------------------------------------------------------------------------

% If you don't need one or more of the below, just remove the content leaving the command, e.g. \cvnumberphone{}

\profilepic{Photo_cv.jpg} % Profile picture

\cvname{Alexandre Poirrier} % Your name
\cvjobtitle{Ingénieur de l'armement en doctorat de cybersécurité à l'\'Ecole polytechnique} % Job title/career

\cvnumberphone{+33 6 05 07 28 62} % Phone number
\cvmail{alexandre.poirrier@polytechnique.org} % Email address
\cvlinkedin{https://www.linkedin.com/in/alexandre-poirrier}

%----------------------------------------------------------------------------------------

\begin{document}


\aboutme{Entré au corps de l'armement en septembre 2019, je poursuis une formation par la recherche en doctorat à l'\'Ecole polytechnique sur le sujet `Sécurité formelle des réseaux zéro-confiance' depuis juillet 2021.}

\skills{
	\skillsection{Programmation}:
	\begin{itemize}
		\item C avec SIMD et optimisations;
		\item Java;
		\item Python : calcul numérique, réseau, machine learning simple;
		\item Web : expérience avec VueJS.
	\end{itemize}

	\skillsection{Hacking}:
	\begin{itemize}
		\item Cryptographie;
		\item Reverse engineering;
		\item Exploitation de vulnérabilités;
		\item Escalade de privilèges sous UNIX.
	\end{itemize}

	\skillsection{Architecture}:
	\begin{itemize}
		\item Docker.
	\end{itemize}

	\skillsection{Langues parlées}:
	\begin{itemize}
		\item Français (maternel);
		\item Anglais (certification C1).
	\end{itemize}
}


\makeprofile % Print the sidebar

%----------------------------------------------------------------------------------------
%	 EXPERIENCE
%----------------------------------------------------------------------------------------
\section{Expérience professionnelle}

\begin{twenty}
	\twentyitem{Juil. 2021}{Présent}{\textbf{Doctorant}, \'Ecole polytechnique}{Palaiseau}{France}{Sécurité formelle des réseaux zéro-confiance.}
	\twentyitem{Avril 2019}{Août 2019}{\textbf{Chercheur stagiaire}, Cisco Systems}{Issy}{France}{Étude et optimisation de VPP, une stack réseau virtuelle écrite pour hardware non spécifiques.}
	\twentyitem{Juin 2018}{Août 2018}{\textbf{Stagiaire Data scientist}, BeeBryte}{Singapour}{}{Conception d'un algorithme de trading pour optimiser la consommation énergétique de bâtiments commerciaux.\\ Apprentissage par renforcement de réseaux de neurones profonds pour optimiser l'algorithme.}
	\twentyitem{Oct. 2017}{Mai 2018}{\textbf{Colleur} en classes préparatoires}{Sceaux}{France}{Interrogations orales pour préparer des élèves de PSI* au concours d'entrée aux grandes écoles.}
	\twentyitem{Sep. 2016}{Mars 2017}{\textbf{Stagiaire ingénieur} au CMRRF de Kerpape}{Ploemeur}{France}{Animation d'un Fablab : apprentissage de la modélisation 3D à des patients handicapés moteurs, pour leur apprendre à créer des aides techniques leur permettant de retrouver une certaine autonomie au quotidien.}
	%\twentyitem{<dates>}{<title>}{<location>}{<description>}
\end{twenty}

%----------------------------------------------------------------------------------------
\section{Formation}

\begin{twenty} % Environment for a list with descriptions
	\twentyitem{2019}{2021}{Master en informatique à ETH Zürich}{Zürich}{Suisse}{Mineure cybersécurité. GPA: 5.69/6.\\ Master thesis dans le groupe de Cryptographie Appliquée `Continuous authentication in Secure Messaging'.}
	\twentyitem{2016}{2019}{Diplôme d'ingénieur à l'École polytechnique}{Palaiseau}{France}{GPA : 3.92/4, Classement : 44\ieme. Mineure cybersécurité.}
	\twentyitem{2013}{2016}{Classes préparatoires Lycée Henri IV}{Paris}{France}{MPSI/MP*, Option informatique}
\end{twenty}


%----------------------------------------------------------------------------------------
%	 AWARDS
%----------------------------------------------------------------------------------------
\section{Prix et publications}

\begin{twentyshort}
	\twentyitemshort{Preprint}{Continuous Authentication in Secure Messaging. Soumission en cours de révision à IEEE S\&P 2022.}
	\twentyitemshort{Prix}{Vainqueur de l'Imagine Cup France 2018. \\ Application web pour aider les personnes sourdes, malentendantes et les étrangers à accéder plus facilement à du contenu culturel grâce à la réalité augmentée.}
\end{twentyshort}


%----------------------------------------------------------------------------------------
%	 OTHER INFORMATION
%----------------------------------------------------------------------------------------
\section{Intérêts}

\begin{twentyshort} % Environment for a short list with no descriptions
	\twentyitemshort{CTF}{Les Capture The Flag (CTF) sont des compétitions de hacking.\\Régulièrement top 50 compétitions mondiales, 10\ieme{} au challenge Brigitte Friang de la DGSE, 5\ieme{} au challenge DGHack 2021 de la DGA.\\\href{https://github.com/apoirrier/CTFs-writeups}{https://github.com/apoirrier/CTFs-writeups}.}
\end{twentyshort}

\end{document} 
