%%%%%%%%%%%%%%%%%%%%%%%%%%%%%%%%%%%%%%%%%
% Twenty Seconds Resume/CV
% LaTeX Template
% Version 1.1 (8/1/17)
%
% This template has been downloaded from:
% http://www.LaTeXTemplates.com
%
% Original author:
% Carmine Spagnuolo (cspagnuolo@unisa.it) with major modifications by 
% Vel (vel@LaTeXTemplates.com)
%
% License:
% The MIT License (see included LICENSE file)
%
%%%%%%%%%%%%%%%%%%%%%%%%%%%%%%%%%%%%%%%%%

%----------------------------------------------------------------------------------------
%	PACKAGES AND OTHER DOCUMENT CONFIGURATIONS
%----------------------------------------------------------------------------------------

\documentclass[a4paper]{twentysecondcv} % a4paper for A4

\usepackage[french]{babel}
\usetikzlibrary{math}
%----------------------------------------------------------------------------------------
%	 PERSONAL INFORMATION
%----------------------------------------------------------------------------------------

% If you don't need one or more of the below, just remove the content leaving the command, e.g. \cvnumberphone{}

\profilepic{Photo_cv.jpg} % Profile picture

\cvname{\huge Alexandre Poirrier} % Your name
\cvjobtitle{} % Job title/career

\cvdate{} % Date of birth
\cvaddress{} % Short address/location, use \newline if more than 1 line is required
\cvnumberphone{+33 6 05 07 28 62} % Phone number
\cvsite{} % Personal website
\cvmail{alexandre.poirrier@polytechnique.edu} % Email address
\cvlinkedin{https://www.linkedin.com/in/alexandre-poirrier}

%----------------------------------------------------------------------------------------

\begin{document}

%----------------------------------------------------------------------------------------
%	 ABOUT ME
%----------------------------------------------------------------------------------------

\aboutme{} % To have no About Me section, just remove all the text and leave \aboutme{}

%----------------------------------------------------------------------------------------
%	 SKILLS
%----------------------------------------------------------------------------------------

% Skill bar section, each skill must have a value between 0 an 6 (float)
\skills{{
	{\textbf{Programmation} :, 
		{\ - C : analyse et création de paquets réseaux, multithreading},
		{\ - C++ : HPC et calculs distribués (MapReduce)},
		{\ - Java : développement d'API, réseaux},
		{\ - Python: numpy, matplotlib, scapy, tensorflow/keras}},
	{\textbf{3D development} :,
		{\ - Modélisation 3D : Blender, Meshmixer},
		{\ - Moteurs 3D : Unity3D}},
	{\textbf{General} :,
		{\ - Office : LaTeX, Microsoft Office, Prezi, Inkscape},
		{\ - OS : Ubuntu, Windows 10}}}
		}

\languages{{{- Français (langue maternelle)},	
			{\ - Anglais (TOEFL score 108)}, 
		{\ - Allemand (scolaire)}}}
%----------------------------------------------------------------------------------------

\makeprofile{fr} % Print the sidebar

%----------------------------------------------------------------------------------------
%	 INTERESTS
%----------------------------------------------------------------------------------------

%----------------------------------------------------------------------------------------
%	 EDUCATION
%----------------------------------------------------------------------------------------
~\\[0.2cm]
\section{Formation}

\begin{twenty} % Environment for a list with descriptions
	\twentyitem{2016-2019}{École polytechnique}{Palaiseau, France}{GPA : 3.92, Classement corps provisoire : $44^e$ \\ Parcours d'approfondissement en cybersecurité\\ Cours : Cryptographie, IoT, Architechture des réseaux et des ordinateurs, Sécurité des réseaux, Hacking \\ Projets :  \\ - Implementation d'attaques sur réseaux (TCP Syn Flood, DNS and DHCP Spoofing, Overflow Exploit, ...). \\ - Load balancing indépendant des serveurs grâce aux Top of Racks et au Segment Routing.}
	\twentyitem{2013-2016}{Classes préparatoires Lycée Henri IV}{Paris, France}{MPSI/MP*, Option informatique}
	\twentyitem{2013}{Baccalauréat général}{Paris, France}{Lycée Henri IV, Série S, mention bien}
\end{twenty}


%----------------------------------------------------------------------------------------
%	 AWARDS
%----------------------------------------------------------------------------------------
~\\[0.2cm]
\section{Prix et distinctions}

\begin{twenty} % Environment for a short list with no descriptions
	\twentyitem{2018}{Vainqueur de la finale française de la Microsoft Imagine Cup}{France}{L'Imagine Cup est une compétition étudiante qui récompense les projets les plus innovants. \\ Mon équipe a présenté une application web pour aider les personnes sourdes, malentendantes et les étrangers à accéder plus facilement à du contenu culturel grâce à la réalité augmentée. \\ Un brevet sur l'algorithme principal est en cours de dépôt.}
\end{twenty}

%----------------------------------------------------------------------------------------
%	 EXPERIENCE
%----------------------------------------------------------------------------------------
~\\[0.15cm]
\section{Expérience}

\begin{twenty} % Environment for a list with descriptions
	\twentyitem{Avril 2019 -\\Août 2019}{\textbf{Chercheur (stage)}, Cisco Systems}{France}{Étude et optimisation de VPP, une stack réseau virtuelle écrite pour hardware non spécifiques.}
	\twentyitem{Juin 2018 -\\ Août 2018}{\textbf{Stagiaire Data scientist} chez BeeBryte}{Singapour}{Conception d'un algorithme de trading pour optimiser la consommation énergétique de bâtiments commerciaux. \\ Apprentissage par renforcement de réseaux de neurones profonds pour optimiser l'algorithme.}
	\twentyitem{Oct. 2017 -\\ Mai 2018}{\textbf{Colleur} en classes préparatoires}{Sceaux, France}{Interrogations orales pour préparer des élèves de PSI* au concours d'entrée aux grandes écoles.}
	\twentyitem{Sep. 2016 -\\ Mars 2017}{\textbf{Stagiaire ingénieur} au CMRRF de Kerpape}{Ploemeur, France}{Animation d'un Fablab : apprentissage de la modélisation 3D à des patients handicapés moteurs, pour les aider à créer des aides techniques qui les aident à retrouver une certaine autonomie au quotidien. \\ Expérience immersive in dans un centre de rééductation au contact de personnes en situation de handicap et du personnel soignant.}
	%\twentyitem{<dates>}{<title>}{<location>}{<description>}
\end{twenty}

%----------------------------------------------------------------------------------------
%	 OTHER INFORMATION
%----------------------------------------------------------------------------------------
~\\[0.1cm]
\section{Intérêts}

\begin{twentyshort} % Environment for a short list with no descriptions
	\twentyitemshort{Sport}{Escrime, compétitions étudiantes}
	\twentyitemshort{Associations}{Trésorier de l'association (binet) escrime artistique \\ Président du binet magie}
\end{twentyshort}

%----------------------------------------------------------------------------------------
%	 SECOND PAGE EXAMPLE
%----------------------------------------------------------------------------------------

%\newpage % Start a new page

%\makeprofile % Print the sidebar

%\section{Other information}

%\subsection{Review}

%Alice approaches Wonderland as an anthropologist, but maintains a strong sense of noblesse oblige that comes with her class status. She has confidence in her social position, education, and the Victorian virtue of good manners. Alice has a feeling of entitlement, particularly when comparing herself to Mabel, whom she declares has a ``poky little house," and no toys. Additionally, she flaunts her limited information base with anyone who will listen and becomes increasingly obsessed with the importance of good manners as she deals with the rude creatures of Wonderland. Alice maintains a superior attitude and behaves with solicitous indulgence toward those she believes are less privileged.

%\section{Other information}

%\subsection{Review}

%Alice approaches Wonderland as an anthropologist, but maintains a strong sense of noblesse oblige that comes with her class status. She has confidence in her social position, education, and the Victorian virtue of good manners. Alice has a feeling of entitlement, particularly when comparing herself to Mabel, whom she declares has a ``poky little house," and no toys. Additionally, she flaunts her limited information base with anyone who will listen and becomes increasingly obsessed with the importance of good manners as she deals with the rude creatures of Wonderland. Alice maintains a superior attitude and behaves with solicitous indulgence toward those she believes are less privileged.

%----------------------------------------------------------------------------------------

\end{document} 
